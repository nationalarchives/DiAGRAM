%------------------------------------------------------------------------------
% This template was created by taking a copy of rmd/latex/default-1.17.0.2.tex,
% removing the bits we don't need (to make it more readable) and then
% shoehorning our code into it.
%------------------------------------------------------------------------------

%------------------------------------------------------------------------------
\documentclass[a4paper,oneside]{article}
%------------------------------------------------------------------------------

%------------------------------------------------------------------------------
% Packages

% Rotate pages
\usepackage{pdflscape}
% Set font size (default to 12pt)
\usepackage[fontsize=12pt]{scrextend}
% Use the lato font (everywhere)
\usepackage[default]{lato}
% Used for colours
\usepackage[table]{xcolor}
% Style sections
\usepackage{sectsty}
% Format sections
\usepackage{titlesec}
% Headers and footers
\usepackage{fancyhdr}
% Use to centre image in footer
\usepackage[export]{adjustbox}
% Customise lists
\usepackage{enumitem}
% Maths stuff
\usepackage{amssymb,amsmath}
% Detect compiler
\usepackage{ifxetex,ifluatex}
% Modern latex stuff
\usepackage{etoolbox}
% Hyperlinks
\usepackage{hyperref}
% Tables
\usepackage{longtable,booktabs,cellspace,makecell}
% Columns
\usepackage{multicol}
% Format captions of figures
\usepackage[labelfont={bf}]{caption}
% Line spacing
\usepackage[onehalfspacing]{setspace}
% Generate dummy text
\usepackage{lipsum}
% Add background footer
\usepackage{eso-pic}
% Vertical spacing between paragraphs
\usepackage{parskip}
% Center all images by default
\usepackage{floatrow}
% Format output of \today
\usepackage[ddmmyyyy]{datetime}
% Reset page colour after title
\usepackage{afterpage}
% Stretch tables
\usepackage{array}
% Graphics stuff
\usepackage{graphicx,grffile}
% Font lyf
\usepackage{fontspec}
%------------------------------------------------------------------------------

%------------------------------------------------------------------------------
% https://stackoverflow.com/a/47122900
%------------------------------------------------------------------------------

%------------------------------------------------------------------------------
\newfontfamily\supria[
  Path = /home/theo/R/x86_64-pc-linux-gnu-library/3.5/diagramNAT/rmarkdown/templates/nata-report/resources/,
  BoldFont=Supria-Sans-Condensed-Bold.otf,
  ]
{Supria-Sans-Condensed-Regular.otf}

% For titlepage
\newfontfamily\bigsupria[
  Path = /home/theo/R/x86_64-pc-linux-gnu-library/3.5/diagramNAT/rmarkdown/templates/nata-report/resources/,
  Scale=6]
{Supria-Sans-Condensed-Regular.otf}

\newfontfamily\opensans[
  Path = /home/theo/R/x86_64-pc-linux-gnu-library/3.5/diagramNAT/rmarkdown/templates/nata-report/resources/
]
{OpenSans-Regular.ttf}

\newfontfamily\roboto[
  Path = /home/theo/R/x86_64-pc-linux-gnu-library/3.5/diagramNAT/rmarkdown/templates/nata-report/resources/
]
{RobotoMono-Regular.ttf}

\setmainfont[
  Path = /home/theo/R/x86_64-pc-linux-gnu-library/3.5/diagramNAT/rmarkdown/templates/nata-report/resources/,
  ]{RobotoMono-Regular.ttf}
%------------------------------------------------------------------------------

%------------------------------------------------------------------------------
% use upquote if available, for straight quotes in verbatim environments
\IfFileExists{upquote.sty}{\usepackage{upquote}}{}
%------------------------------------------------------------------------------

%------------------------------------------------------------------------------
% use microtype if available
\IfFileExists{microtype.sty}{%
\usepackage{microtype}
\UseMicrotypeSet[protrusion]{basicmath} % disable protrusion for tt fonts
}{}
%------------------------------------------------------------------------------

%------------------------------------------------------------------------------
% Set custom geometry
\usepackage[a4paper,
           top=3cm,
           bottom=2cm,
           right=2cm,
           left=2cm,
           footskip=2cm,
           includefoot]{geometry}
%------------------------------------------------------------------------------

%------------------------------------------------------------------------------
% New column type to align in fixed column width
\newcolumntype{L}[1]{>{\raggedleft\arraybackslash}m{#1}}
\newcolumntype{C}[1]{>{\centering\arraybackslash}m{#1}}

% Make table in header bigger
\renewcommand{\arraystretch}{1.75}

\pagestyle{fancy}

% Remove default hrule
\renewcommand{\headrulewidth}{0pt}

% Clear out headers and footers
\fancyhead{}
\fancyfoot{}

\newlength{\footersize}
\setlength{\footersize}{0.9\textwidth}

% Header for landscape pages
\fancyfoot[C]{%
    \roboto
    \begin{tabular}{|C{0.15\footersize}|C{0.25\footersize}|L{0.59\footersize}|C{0.01\footersize}|}
    \hline
    \centering\rotatebox[origin=c]{180}{\today} &
    \rotatebox[origin=c]{180}{\uppercase{DiAGRAM}} &
    \rotatebox[origin=c]{180}{\quad{}The National Archives} &
    \rotatebox[origin=c]{90}{\centering\,\thepage}\\
    \hline
    \end{tabular}%
}
%------------------------------------------------------------------------------

%------------------------------------------------------------------------------
% Graphics stuff
\makeatletter
\def\maxwidth{\ifdim\Gin@nat@width>\linewidth\linewidth\else\Gin@nat@width\fi}
\def\maxheight{\ifdim\Gin@nat@height>\textheight\textheight\else\Gin@nat@height\fi}
\makeatother

% Scale images is necessary, so that they will not overflow the page
% margins by default, and it is still possible to overwrite the defaults
% using explicit options in \includegraphics[width, height, ...]{}
\setkeys{Gin}{width=\maxwidth,height=\maxheight,keepaspectratio}
%------------------------------------------------------------------------------

%------------------------------------------------------------------------------
% Don't indent paragraphs
\setlength{\parindent}{0pt}
%------------------------------------------------------------------------------

%------------------------------------------------------------------------------
% Section numbering
\setcounter{secnumdepth}{0}
%------------------------------------------------------------------------------

%------------------------------------------------------------------------------
% Redefines (sub)paragraphs to behave more like sections
\ifx\paragraph\undefined\else
\let\oldparagraph\paragraph
\renewcommand{\paragraph}[1]{\oldparagraph{#1}\mbox{}}
\fi
\ifx\subparagraph\undefined\else
\let\oldsubparagraph\subparagraph
\renewcommand{\subparagraph}[1]{\oldsubparagraph{#1}\mbox{}}
\fi
%------------------------------------------------------------------------------

%------------------------------------------------------------------------------
%%% Use protect on footnotes to avoid problems with footnotes in titles
\let\rmarkdownfootnote\footnote%
\def\footnote{\protect\rmarkdownfootnote}
%------------------------------------------------------------------------------

%------------------------------------------------------------------------------
% Define nata colours
\definecolor{nataBlack}{HTML}{26262A}
\definecolor{nataCoolGrey}{HTML}{D9D9D6}
\definecolor{nataYellow}{HTML}{F9F7E2}
\definecolor{nataPink}{HTML}{FAD3D4}
\definecolor{nataGrey}{HTML}{8C9694}
\definecolor{nataBrown}{HTML}{B2A38E}
\definecolor{nataOrange}{HTML}{F9E1BC}
\definecolor{nataOlive}{HTML}{DDE5D5}
%------------------------------------------------------------------------------

%------------------------------------------------------------------------------
% Colour the font
\color{nataBlack}
%------------------------------------------------------------------------------

%------------------------------------------------------------------------------
% Format sections
\sectionfont{\huge\color{nataBlack}\supria\uppercase}
\subsectionfont{\Large\color{nataBlack}\supria\uppercase}
\subsubsectionfont{\large\color{nataBlack}\supria}
%------------------------------------------------------------------------------

%------------------------------------------------------------------------------
% Style the title page

% Style title
\makeatletter
\def\@maketitle{%
    \newpage
    \pagecolor{nataBlack}
    \null
    \vskip 2cm%
    \color{white}\bigsupria{DiAGRAM}\\
    \supria\Huge{Digital Archiving Graphical Risk Assessment Model}\\
    \vfill\hfill
    \includegraphics[width=0.6\textwidth]{/home/theo/R/x86_64-pc-linux-gnu-library/3.5/diagramNAT/rmarkdown/templates/nata-report/resources/tna-logo.png}
    }
\makeatother

% Don't include a page number on the first page
\let\oldmaketitle\maketitle
\renewcommand{\maketitle}{\oldmaketitle\thispagestyle{empty}\clearpage}
%------------------------------------------------------------------------------

%------------------------------------------------------------------------------
% Style lists
\providecommand{\tightlist}{%
    \setlength{\itemsep}{2pt}\setlength{\parskip}{0pt}}

\setlist[itemize]{topsep=5pt}

\setlist[itemize,1]{label=\color{nataBlack}\textbullet}
\setlist[itemize,2]{label={\color{nataBlack}\bf\textendash}}
\setlist[itemize,3]{label={\color{nataBlack}\bf\textasteriskcentered}}
%------------------------------------------------------------------------------

%------------------------------------------------------------------------------
% Extra includes
%------------------------------------------------------------------------------

%------------------------------------------------------------------------------
% Make tables sensible
\renewcommand{\arraystretch}{1.25}
%------------------------------------------------------------------------------

%------------------------------------------------------------------------------
% Main document
\begin{document}
\begin{landscape}


\begin{titlepage}
\maketitle
\end{titlepage}

\pagecolor{nataCoolGrey}

\hypertarget{question-1-digital-object}{%
\subsection{Question 1: Digital Object}\label{question-1-digital-object}}

\textbf{Definition}: The proportion of your archive made up of born-digital, digitised and surrogate files.\\
\textbf{Explanation}: Different types of digital material hold different risks, for example some file formats may be easier to preserve than others. The amount of metadata held about the material and their conditions of use will also differ. All these aspects contribute to their digital preservation risk.
If your archive does not distinguish between surrogates and digitised records put the percentage for surrogates only.

What proportion of your digital archive are the following?

\begin{itemize}
\tightlist
\item
  Born Digital - Records were created in a digital format.\\
\item
  Digitised - Records have been created as a result of converting analogue originals, but you do not hold those originals.\\
\item
  Surrogate - Digital images have been created as a result of converting analogue originals, and you also hold the originals.
\end{itemize}

\newpage

\hypertarget{question-2-storage-medium}{%
\subsection{Question 2: Storage Medium}\label{question-2-storage-medium}}

\textbf{Definition}: The type of media on which your digital material is stored such as USB hard drives, CDs or the Cloud.\\
\textbf{Explanation}: Storage medium makes a big difference to the longevity of your digital material. Some are designed to be very robust (some degree of error detection and correction built in or less susceptible to many errors by design) and others are not intended for long-term storage.\\
What proportion of your records are stored on the following media types?

\begin{itemize}
\tightlist
\item
  A Less stable - Expected lifespan below 10 years or unknown, highly susceptible to physical damage, requires specific environmental conditions and very sensitive to changes, does not support error-detection methods, supporting technology is novel, proprietary and limited. Examples include USB flash drives (memory sticks), floppy disks, SD drives and CD-R discs.\\
\item
  B More stable - A proven lifespan of at least 10 years, low susceptibility to physical damage, tolerant of a wide range of environmental conditions without data loss, supports robust error-detection methods, supporting technology is well established and widely available. Examples include LTO tapes, Blu-ray discs, enterprise/corporate managed hard drives and CD-ROM discs.\\
\item
  C Outsourced Data Storage - An external company is responsible for our digital storage. Examples include Amazon Simple Storage Service, Microsoft Azure Archive Storage and Google Cloud Storage.
\end{itemize}

\newpage

\hypertarget{question-3.1-rep-and-refresh}{%
\subsection{Question 3.1: Rep and Refresh}\label{question-3.1-rep-and-refresh}}

\textbf{Definition}: Your archive's policies on making copies and regularly moving digital material on to newer versions of the storage media.

\textbf{Explanation}: It is good practice to keep more than one copy of your digital material. If you do not do this for any material answer 0\%.

What percentage of your material do you have at least one additional copy of?

\newpage

\hypertarget{question-3.2-rep-and-refresh}{%
\subsection{Question 3.2: Rep and Refresh}\label{question-3.2-rep-and-refresh}}

\textbf{Definition}: Your archive's policies on making copies and regularly moving digital material on to newer versions of the storage media.

\textbf{Explanation}: As well as keeping copies of your digital material, it is good practice to refresh your storage media (i.e.~move the material on to newer versions of the LTO tapes or hard drives for example). This reduces the risk that the material is not corrupted as storage media ages and requires replacement.

For those files with an additional copy, do you ensure you always have at least 2 independent copies?

\newpage

\hypertarget{question-4.1-op-environment}{%
\subsection{Question 4.1: Op Environment}\label{question-4.1-op-environment}}

\textbf{Definition}: Your archive's policy on the storage location of your digital material.

\textbf{Explanation}: It is good practice to keep a copy of your digital material offsite in case of a disaster at your primary location.

What percentage of your digital material has a copy kept offsite?

\newpage

\hypertarget{question-4.2-op-environment}{%
\subsection{Question 4.2: Op Environment}\label{question-4.2-op-environment}}

\textbf{Definition}: Your archive's policy on the storage location of your digital material.

\textbf{Explanation}: In the current version of the tool, the operating environment variable is only affected by the physical disaster of flood (not for example by fire or earthquake) so we are only asking if you have protection in place for flood. Other types of physical disasters may be added to later versions of the tool.

If all of your digital material is in one location, is there adequate protection against damage from a flood?

\begin{itemize}
\tightlist
\item
  Yes\\
\item
  No\\
\item
  Not Applicable - we have copies offsite
\end{itemize}

\newpage

\hypertarget{question-5-physical-disaster}{%
\subsection{Question 5: Physical Disaster}\label{question-5-physical-disaster}}

\textbf{Definition}: The risk of a flood at your archive's primary storage location.

\textbf{Explanation}: Where the risk level varies for different risk types (i.e.~flood risks from rivers or the sea, flood risks from surface water, flood risks from reservoirs), please answer based on the highest result.
Note that for this first version of DiAGRAM, flood is the only physical disaster included, as this is the most likely physical disaster in the UK.

Based on the Government's long term flood risk assessment, how likely is it that your safest digital storage location will experience a flood?
Click here to check your flood risk: https://flood-warning-information.service.gov.uk/long-term-flood-risk/postcode

\begin{itemize}
\tightlist
\item
  Very Low\\
\item
  Low\\
\item
  Medium\\
\item
  High
\end{itemize}

\newpage

\hypertarget{question-6-checksum}{%
\subsection{Question 6: Checksum}\label{question-6-checksum}}

\textbf{Definition}: A unique numerical signature dreived from a file that can be used to compare copies Definition from the DPC handbook. A checksum is needed to ensure integrity of the digital object.

\textbf{Explanation}: A checksum is needed to ensure the integrity of the digital object. Some depositors are unable to include checksums for the material they deposit in the archive. In these cases, archivists may decide to generate checksums for the material when they receive it to enable them to check it has not changed while in their custody.
If your digital material does not have checksums or the checksums are generated after it is accessioned, choose 100\% for C.

For what proportion of files do you have a checksum from following sources?

\begin{itemize}
\tightlist
\item
  TRUE - The depositor\\
\item
  Archivist-generated - The archivist, generated on receipt of the record but prior to accessioning\\
\item
  FALSE - You don't have checksums at all, or they were generated sometime after initial receipt
\end{itemize}

\newpage

\hypertarget{question-7.1-system-security}{%
\subsection{Question 7.1: System Security}\label{question-7.1-system-security}}

\textbf{Definition}: A secure system can protect data from deletion or modification from any unauthorized party, and it ensures that when an authorized person makes a change that should not have been made, the damage can be reversed. Definition from Forcepoint, https://www.forcepoint.com/cyber-edu/cia-triad.

\textbf{Explanation}: Many archives do not have direct control of the security of their archival systems. If this is your situation, you can answer for the security managed at a corporate level.

Does your organisation hold a recognised security accreditation such as Cyber Essentials or ISO 27001 (or has it carried out equivalent assessments)?

\begin{itemize}
\tightlist
\item
  FALSE\\
\item
  Cyber Essentials\\
\item
  Cyber Essentials Plus\\
\item
  ISO 27001
\end{itemize}

\newpage

\hypertarget{question-7.2-system-security}{%
\subsection{Question 7.2: System Security}\label{question-7.2-system-security}}

\textbf{Definition}: A secure system can protect data from deletion or modification from any unauthorized party, and it ensures that when an authorized person makes a change that should not have been made, the damage can be reversed. Definition from Forcepoint, https://www.forcepoint.com/cyber-edu/cia-triad.

\textbf{Explanation}: If you don't know what a penetration (or ``Pen'') test is then choose ``No test''.

Have your archival systems had a penetration test? If yes, are any issues outstanding?

\begin{itemize}
\tightlist
\item
  No test\\
\item
  Critical issues outstanding\\
\item
  Severe issues outstanding\\
\item
  None, or only minor issues outstanding
\end{itemize}

\newpage

\hypertarget{question-7.3-system-security}{%
\subsection{Question 7.3: System Security}\label{question-7.3-system-security}}

\textbf{Definition}: A secure system can protect data from deletion or modification from any unauthorized party, and it ensures that when an authorized person makes a change that should not have been made, the damage can be reversed. Definition from Forcepoint, https://www.forcepoint.com/cyber-edu/cia-triad.

\textbf{Explanation}: By asking you to assess your skills against a standard, the tool will be able to give a more objective score for this question.

Referring to the NDSA Levels of Preservation, what level is your archive for the Control functional area?

\begin{itemize}
\tightlist
\item
  Not achieved\\
\item
  Level 1\\
\item
  Level 2\\
\item
  Level 3\\
\item
  Level 4
\end{itemize}

\newpage

\hypertarget{question-7.4-system-security}{%
\subsection{Question 7.4: System Security}\label{question-7.4-system-security}}

\textbf{Definition}: A secure system can protect data from deletion or modification from any unauthorized party, and it ensures that when an authorized person makes a change that should not have been made, the damage can be reversed. Definition from Forcepoint, https://www.forcepoint.com/cyber-edu/cia-triad.

\textbf{Explanation}: If the result is not recorded choose No as you cannot prove that the file was virus-free when it was received.

Is all of your digital material virus checked and the result recorded?

\begin{itemize}
\tightlist
\item
  FALSE\\
\item
  TRUE
\end{itemize}

\newpage

\hypertarget{question-8.1-info-management}{%
\subsection{Question 8.1: Info Management}\label{question-8.1-info-management}}

\textbf{Definition}: Internal systems and support for coherent information management and documentation of preservation actions. This is needed to ensure integrity and provenance of the digital object.

\textbf{Explanation}: For this tool, information management systems refers to the recording of digital preservation activities such as emulation, copying and fixity checking. It does not refer to your online catalogue or the broader information management capabilities of your wider organisation.
By using existing standards we will get more consistent answers to the questions, so answers will be more comparable between different occasions that you use the tool

Referring to the NDSA Levels of Preservation, what level is your archive for the Metadata functional area?

\begin{itemize}
\tightlist
\item
  Not achieved\\
\item
  Level 1\\
\item
  Level 2\\
\item
  Level 3\\
\item
  Level 4
\end{itemize}

\newpage

\hypertarget{question-8.2-info-management}{%
\subsection{Question 8.2: Info Management}\label{question-8.2-info-management}}

\textbf{Definition}: Internal systems and support for coherent information management and documentation of preservation actions. This is needed to ensure integrity and provenance of the digital object.

\textbf{Explanation}: For this tool, information management refers to the recording of digital preservation activities such as emulation, copying and fixity checking. It does not refer to the information management capabilities of your wider organisation such as a library, university or business.

Referring to the NDSA Levels of Preservation, what level is your archive for the Content functional area?

\begin{itemize}
\tightlist
\item
  Not achieved\\
\item
  Level 1\\
\item
  Level 2\\
\item
  Level 3\\
\item
  Level 4
\end{itemize}

\newpage

\hypertarget{question-8.3-info-management}{%
\subsection{Question 8.3: Info Management}\label{question-8.3-info-management}}

\textbf{Definition}: Internal systems and support for coherent information management and documentation of preservation actions. This is needed to ensure integrity and provenance of the digital object.

\textbf{Explanation}: By asking you to assess your skills against a standard, the tool will be able to give a more objective score for this question.

Referring to DPC RAM, what level is your archive for Service capability, I - Content preservation\\
Referring to DPC RAM, what level is your archive for Service capability, J - Metadata management

\begin{itemize}
\tightlist
\item
  Minimal awareness\\
\item
  Awareness\\
\item
  Basic\\
\item
  Managed\\
\item
  Optimized
\end{itemize}

\newpage

\hypertarget{question-9-technical-skills}{%
\subsection{Question 9: Technical Skills}\label{question-9-technical-skills}}

\textbf{Definition}: Bespoke digital preservation skills such as awareness of technological trends, detailed knowledge of storage media, hardware and software, skills to perform file format migration, skills to find emulating software etc?\\
\textbf{Explanation}: By asking you to assess your skills against a standard, the tool will be able to give a more objective score for this question.

KIA 1.9 Apply appropriate technological solutions\\
KIA 1.12 Digital preservation standards\\
KIA 1.15 Information technology definitions and skills\\
KIA 1.16 Select and apply digital curation and preservation techniques\\
KIA 3.4 Continuously monitor and evaluate digital curation technologies\\
KIA 5.1 Data structures and types\\
KIA 5.2 File types, applications and systems\\
KIA 5.3 Database types and structures\\
KIA 5.4 Execute analysis of and forensic procedures in digital curation\\
PQ 3.9 Translate current digital curation knowledge into new services and tools

\begin{itemize}
\tightlist
\item
  None\\
\item
  Basic\\
\item
  Intermediate\\
\item
  Advanced
\end{itemize}

\newpage

\end{landscape}
\end{document}
%------------------------------------------------------------------------------

